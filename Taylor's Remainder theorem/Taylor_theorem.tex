\documentclass{article}
\usepackage{kotex}
\usepackage{amssymb}
\usepackage{amsmath}
\usepackage{setspace}
\begin{document}
\setstretch{1.6}
\section{테일러의 나머지 정리}
$x=a$ 를 포함하는 개구간 $I$에서 함수 $f$가 $n+1$번 미분 가능할 때, 모든 양의 정수 $n$과 모든 $x \in \mathbb{R}$에 대하여
\begin{equation}
f(x)=f(a)+f'(a)(x-a)+\cfrac{f''(a)}{2!}(x-a)^2+\dots+\cfrac{f^{(n)}(a)}{n!}(x-a)^n+R_n(x)
\end{equation}
이때, $R_n(x)=\cfrac{f^{(n+1)}(c)}{n+1!}(x-a)^{(n+1)}$ 을 만족하는 $c$가 $a<c<x$ 안에 적어도 하나 존재한다.

\section{증명}
일반성을 잃지 않고 $a<x$라 하자.
\begin{equation}
f(x)=f(a)+f'(a)(x-a)+\cfrac{f''(a)}{2!}(x-a)^2+\dots+\cfrac{f^{(n)}(a)}{n!}(x-a)^n+M(x-a)^{n+1}
\end{equation}
이라 정의할 때, $M=\cfrac{f^{(n+1)}(c)}{n+1!}$ 임을 보이는 것으로 충분하다.\newline
모든 $x \in I$에 대하여 $x\leq b$을 만족하는 실수 $b$를 잡자. 그리고 $g : [a,b] \longrightarrow \mathbb{R}$의 정의역 $[a,b]$에 속하는 임의의 실수 $t$에 대하여, 대응되는 치역의 원소 $g(t)$를 다음과 같이 정의한다.
\begin{equation}
g(t)=f(t)+f'(t)(x-t)+\cfrac{f''(t)}{2!}(x-t)^2+\dots+\cfrac{f^{(n)}(a)}{n!}(x-a)^n+M(x-t)^{n+1}
\end{equation}
문제의 가정으로부터 $I$에서 정의된 함수 $f$는 $n+1$번 미분 가능하고 폐구간 $[a,b]\subseteq I$이므로 $g(t)$ 또한 $[a,b]$에서 연속이고 개구간 $(a,b)$에서 미분 가능하다.\newline
그런데 $[a,x] \subseteq [a,b]$이므로 $g(t)$가 $[a,x]$에서 연속이고 $(a,x)$에서 미분 가능하다. $g(a)=g(x)$를 만족할 경우, $g(t)$에 롤의 정리를 적용할 수 있다. \newline
\begin{equation}
g(a)=f(a)=g(x)
\end{equation}
따라서, 롤의 정리에 의하여 $g'(c)=0$이 되는 실수 $c$가 개구간 $(a,x)$안에 적어도 하나 반드시 존재한다. \newline
이제, $g(t)$를 독립변수 $t$에 대해 미분하면:
\begin{eqnarray}
g'(t)=f'(t)+f''(t)(x-t)-f'(t)+\cfrac{f'''(t)}{2!}(x-t)^2-f''(t)(x-t)+\dots
\end{eqnarray}
\begin{equation}
+\dots \cfrac{f^{(n+1)}(t)}{n!}(x-t)^n-n\cfrac{f^{(n)}(t)}{n!}(x-t)^{n-1}-M(n+1)(x-t)^n
\end{equation}
이는 망원급수(telescoping series)로서 유한개의 항을 재배열하면 부분적 항들의 합이 소거되므로 다음과 같이 정리된다.
\begin{equation}
\cfrac{f^{(n+1)}(t)}{n!}(x-t)^n-M(n+1)(x-t)^n
\end{equation}
롤의 정리에 의해 $g'(c)=0$이 되는 실수 $c$가 존재하므로 $c$를 대입하고 $M$에 대하여 정리하면 다음을 얻는다.
\begin{equation}
M(x-c)^n=\cfrac{f^{(n+1)}(c)}{n!}(x-c)^n
\end{equation}
문제의 가정으로부터 $c$는 $a<c<x$안에 존재하므로 $x-c \neq0$, 양변을 $(x-c)^n$으로 나누면:
\begin{equation}
M=\cfrac{f^{(n+1)}(c)}{n+1!}
\end{equation}
최종적으로 원하는 답을 얻는다.
\end{document}